% Options for packages loaded elsewhere
\PassOptionsToPackage{unicode}{hyperref}
\PassOptionsToPackage{hyphens}{url}
%
\documentclass[
]{article}
\usepackage{amsmath,amssymb}
\usepackage{lmodern}
\usepackage{iftex}
\ifPDFTeX
  \usepackage[T1]{fontenc}
  \usepackage[utf8]{inputenc}
  \usepackage{textcomp} % provide euro and other symbols
\else % if luatex or xetex
  \usepackage{unicode-math}
  \defaultfontfeatures{Scale=MatchLowercase}
  \defaultfontfeatures[\rmfamily]{Ligatures=TeX,Scale=1}
\fi
% Use upquote if available, for straight quotes in verbatim environments
\IfFileExists{upquote.sty}{\usepackage{upquote}}{}
\IfFileExists{microtype.sty}{% use microtype if available
  \usepackage[]{microtype}
  \UseMicrotypeSet[protrusion]{basicmath} % disable protrusion for tt fonts
}{}
\makeatletter
\@ifundefined{KOMAClassName}{% if non-KOMA class
  \IfFileExists{parskip.sty}{%
    \usepackage{parskip}
  }{% else
    \setlength{\parindent}{0pt}
    \setlength{\parskip}{6pt plus 2pt minus 1pt}}
}{% if KOMA class
  \KOMAoptions{parskip=half}}
\makeatother
\usepackage{xcolor}
\usepackage[margin=1in]{geometry}
\usepackage{color}
\usepackage{fancyvrb}
\newcommand{\VerbBar}{|}
\newcommand{\VERB}{\Verb[commandchars=\\\{\}]}
\DefineVerbatimEnvironment{Highlighting}{Verbatim}{commandchars=\\\{\}}
% Add ',fontsize=\small' for more characters per line
\usepackage{framed}
\definecolor{shadecolor}{RGB}{248,248,248}
\newenvironment{Shaded}{\begin{snugshade}}{\end{snugshade}}
\newcommand{\AlertTok}[1]{\textcolor[rgb]{0.94,0.16,0.16}{#1}}
\newcommand{\AnnotationTok}[1]{\textcolor[rgb]{0.56,0.35,0.01}{\textbf{\textit{#1}}}}
\newcommand{\AttributeTok}[1]{\textcolor[rgb]{0.77,0.63,0.00}{#1}}
\newcommand{\BaseNTok}[1]{\textcolor[rgb]{0.00,0.00,0.81}{#1}}
\newcommand{\BuiltInTok}[1]{#1}
\newcommand{\CharTok}[1]{\textcolor[rgb]{0.31,0.60,0.02}{#1}}
\newcommand{\CommentTok}[1]{\textcolor[rgb]{0.56,0.35,0.01}{\textit{#1}}}
\newcommand{\CommentVarTok}[1]{\textcolor[rgb]{0.56,0.35,0.01}{\textbf{\textit{#1}}}}
\newcommand{\ConstantTok}[1]{\textcolor[rgb]{0.00,0.00,0.00}{#1}}
\newcommand{\ControlFlowTok}[1]{\textcolor[rgb]{0.13,0.29,0.53}{\textbf{#1}}}
\newcommand{\DataTypeTok}[1]{\textcolor[rgb]{0.13,0.29,0.53}{#1}}
\newcommand{\DecValTok}[1]{\textcolor[rgb]{0.00,0.00,0.81}{#1}}
\newcommand{\DocumentationTok}[1]{\textcolor[rgb]{0.56,0.35,0.01}{\textbf{\textit{#1}}}}
\newcommand{\ErrorTok}[1]{\textcolor[rgb]{0.64,0.00,0.00}{\textbf{#1}}}
\newcommand{\ExtensionTok}[1]{#1}
\newcommand{\FloatTok}[1]{\textcolor[rgb]{0.00,0.00,0.81}{#1}}
\newcommand{\FunctionTok}[1]{\textcolor[rgb]{0.00,0.00,0.00}{#1}}
\newcommand{\ImportTok}[1]{#1}
\newcommand{\InformationTok}[1]{\textcolor[rgb]{0.56,0.35,0.01}{\textbf{\textit{#1}}}}
\newcommand{\KeywordTok}[1]{\textcolor[rgb]{0.13,0.29,0.53}{\textbf{#1}}}
\newcommand{\NormalTok}[1]{#1}
\newcommand{\OperatorTok}[1]{\textcolor[rgb]{0.81,0.36,0.00}{\textbf{#1}}}
\newcommand{\OtherTok}[1]{\textcolor[rgb]{0.56,0.35,0.01}{#1}}
\newcommand{\PreprocessorTok}[1]{\textcolor[rgb]{0.56,0.35,0.01}{\textit{#1}}}
\newcommand{\RegionMarkerTok}[1]{#1}
\newcommand{\SpecialCharTok}[1]{\textcolor[rgb]{0.00,0.00,0.00}{#1}}
\newcommand{\SpecialStringTok}[1]{\textcolor[rgb]{0.31,0.60,0.02}{#1}}
\newcommand{\StringTok}[1]{\textcolor[rgb]{0.31,0.60,0.02}{#1}}
\newcommand{\VariableTok}[1]{\textcolor[rgb]{0.00,0.00,0.00}{#1}}
\newcommand{\VerbatimStringTok}[1]{\textcolor[rgb]{0.31,0.60,0.02}{#1}}
\newcommand{\WarningTok}[1]{\textcolor[rgb]{0.56,0.35,0.01}{\textbf{\textit{#1}}}}
\usepackage{graphicx}
\makeatletter
\def\maxwidth{\ifdim\Gin@nat@width>\linewidth\linewidth\else\Gin@nat@width\fi}
\def\maxheight{\ifdim\Gin@nat@height>\textheight\textheight\else\Gin@nat@height\fi}
\makeatother
% Scale images if necessary, so that they will not overflow the page
% margins by default, and it is still possible to overwrite the defaults
% using explicit options in \includegraphics[width, height, ...]{}
\setkeys{Gin}{width=\maxwidth,height=\maxheight,keepaspectratio}
% Set default figure placement to htbp
\makeatletter
\def\fps@figure{htbp}
\makeatother
\setlength{\emergencystretch}{3em} % prevent overfull lines
\providecommand{\tightlist}{%
  \setlength{\itemsep}{0pt}\setlength{\parskip}{0pt}}
\setcounter{secnumdepth}{-\maxdimen} % remove section numbering
\ifLuaTeX
  \usepackage{selnolig}  % disable illegal ligatures
\fi
\IfFileExists{bookmark.sty}{\usepackage{bookmark}}{\usepackage{hyperref}}
\IfFileExists{xurl.sty}{\usepackage{xurl}}{} % add URL line breaks if available
\urlstyle{same} % disable monospaced font for URLs
\hypersetup{
  pdftitle={Installing R and RStudio},
  pdfauthor={Sean Tavares},
  hidelinks,
  pdfcreator={LaTeX via pandoc}}

\title{Installing R and RStudio}
\author{Sean Tavares}
\date{}

\begin{document}
\maketitle

\hypertarget{installing-r-and-rstudio}{%
\section{Installing R and RStudio}\label{installing-r-and-rstudio}}

For this course you'll need to download and install \texttt{R} and
RStudio---all assignments will be using \texttt{R}, including the
homeworks, graphics discussions, and project report.

To download \texttt{R}, follow the instructions
\href{https://cran.rstudio.com/}{here}. Be sure to choose the correct
operating system, and 64-bit R if possible/compatible.

After you've installed \texttt{R}, install RStudio by following the
instructions
\href{https://rstudio.com/products/rstudio/download/\#download}{here}.

If you run into any issues with installing \texttt{R} or RStudio and/or
opening RStudio on your computer, please post on Piazza describing the
issue you're running into, and we'll do our best to help you ASAP.

Once you get RStudio open, download the Demo0.Rmd file from Canvas and
open it in RStudio (File / Open\ldots). When you open the .Rmd file,
you'll notice several panes in RStudio. The most important pane is where
the actual .Rmd file is (i.e., where this text is within RStudio),
because this is where you'll type your answers to construct the final
.Rmd and HTML file that you'll submit for homeworks.

To get some practice writing text in the .Rmd file, scroll to the top of
the .Rmd file where it says ``Your Name Here''. Replace ``Your Name
Here'' with your name (\textbf{you should do this for all assignments}).
After you've done this, look for the ``Knit'' button near the top of
RStudio. Click the down-arrow there and click ``Knit to HTML''. This
should generate an .html file with your name at the top. (This .html
should be in the same place on your computer where your .Rmd file is)
After you click the ``Knit to HTML'' button once, you can just click the
``Knit'' button itself to keep updating the .html file; you can also use
Cmd+Shift+K (Mac) or Ctrl+Shift+K (Windows/Linux).

\textbf{Continue to the next part only after you've successfully changed
the ``Your Name Here'' text and Knitted the .Rmd file to HTML.}

\hypertarget{installing-packages-and-loading-libraries}{%
\section{Installing Packages and Loading
Libraries}\label{installing-packages-and-loading-libraries}}

After you've downloaded \texttt{R} and RStudio, there's already plenty
of functions that you can use. For example, the following code produces
5 random draws from a standard Normal distribution (i.e., the
distribution N(0,1)) and prints out the mean of those draws:

\begin{Shaded}
\begin{Highlighting}[]
\NormalTok{draws }\OtherTok{\textless{}{-}} \FunctionTok{rnorm}\NormalTok{(}\DecValTok{5}\NormalTok{)}
\FunctionTok{mean}\NormalTok{(draws)}
\end{Highlighting}
\end{Shaded}

\begin{verbatim}
## [1] 0.4954754
\end{verbatim}

Check the .Rmd file to see how I included \texttt{R} code within the
.Rmd file.

There are some functions that aren't automatically available in
\texttt{R} but can quickly be made available by loading packages In
\texttt{R}, packages are basically collections of functions. There are
many packages automatically available in \texttt{R}; for example, the
following code uses the \texttt{library()} function to load the
\texttt{MASS} package:

\begin{Shaded}
\begin{Highlighting}[]
\FunctionTok{library}\NormalTok{(MASS)}
\end{Highlighting}
\end{Shaded}

Unfortunately, there are many \texttt{R} packages that are not currently
installed on your computer. For instance, throughout the class we'll
extensively use the
\href{https://www.tidyverse.org/}{\texttt{tidyverse}} suite of packages
- which includes the popular data visualization package
\texttt{ggplot2}. Try running the following line of code at the command
line / Console in RStudio (\textbf{NOT in the .Rmd file}):

\texttt{install.packages("tidyverse")}

For many of you, this should work with no errors; for those of you who
do get an error, we provide further instructions below that may help. If
you are able to successfully install the \texttt{tidyverse} library, the
following line of code should run within the .Rmd file (just delete the
hastag \# and then try to Knit your .Rmd file):

\begin{Shaded}
\begin{Highlighting}[]
\FunctionTok{library}\NormalTok{(tidyverse)}
\end{Highlighting}
\end{Shaded}

\begin{verbatim}
## -- Attaching packages --------------------------------------- tidyverse 1.3.2 --
## v ggplot2 3.3.6     v purrr   0.3.4
## v tibble  3.1.8     v dplyr   1.0.9
## v tidyr   1.2.0     v stringr 1.4.1
## v readr   2.1.2     v forcats 0.5.2
## -- Conflicts ------------------------------------------ tidyverse_conflicts() --
## x dplyr::filter() masks stats::filter()
## x dplyr::lag()    masks stats::lag()
## x dplyr::select() masks MASS::select()
\end{verbatim}

\textbf{Important Note}: NEVER install new packages in a code block in
an .Rmd file. Always install new packages at the command line / Console.
That is, the \texttt{install.packages()} function should NEVER be in
your submitted code. The \texttt{library()} function, however, should be
in most of your submitted code: The \texttt{library()} function loads
packages only after they are installed.

If you're able to successfully run the above line of code, \textbf{you
can skip} to the ``Posting to Gradescope'' section. If you were NOT able
to successfully install the \texttt{tidyverse} using the
\texttt{install.packages()} function, then skip ahead to the section
``Further Steps for Installing Packages''.

\hypertarget{posting-to-gradescope}{%
\section{Posting to Gradescope}\label{posting-to-gradescope}}

If you were able to successfully knit your .Rmd file with the
\texttt{tidyverse} library loaded, you're ready to submit Demo0 to
Gradescope! There's just one more caveat: \textbf{Gradescope only
accepts PDFs and not HTML files}. So, take a moment to convert your HTML
file to PDF by visiting an online file converter like
\url{https://html2pdf.com/}, and then submit the resulting PDF to
Gradescope. Alternatively, you can get RStudio to ``Knit to PDF'', but
you need to install LaTeX on your computer to do this, which isn't
required for the course, but it is slightly more convenient than using
an online file converter. \href{https://www.latex-project.org/get/}{See
here} for installing LaTeX on your computer, if you're interested. (More
generally, LaTeX is a popular software to display mathematical equations
on computers; LaTeX is pronounced ``lah-teck'' or ``lay-teck''.)

\textbf{For ALL assignments, after you make a PDF, always make sure that
your code, graphs, and answers are displayed on your PDF before
submitting it to Gradescope.}

\textbf{Note that your HTML file is in the same place where your .Rmd
file is. So, look for where you downloaded this .Rmd file on your
computer.}

All of the following material is just ``bonus material'' for those
curious about how to format RStudio and RMarkdown files.

\hypertarget{optional-further-steps-for-installing-packages}{%
\section{OPTIONAL: Further Steps for Installing
Packages}\label{optional-further-steps-for-installing-packages}}

\textbf{Remember, this section only applies to you if you were unable to
install and load the \texttt{tidyverse} in the previous section}.

In some cases (e.g., if you're using one of the CMU cluster computers),
the package may not install. This happens because CMU does not allow us
to install new packages to the default location. As a result, we have to
specify a new directory where we can install new \texttt{R} packages.

\textbf{If the \texttt{tidyverse} package installed with no issues, you
can skip the following parts.} If you could not install the package,
take the following steps:

\begin{enumerate}
\def\labelenumi{\alph{enumi}.}
\tightlist
\item
  Create a new directory (i.e., folder) on your computer called
  ``36-613'', and create a new sub-directory called ``packages''. The
  filepath to this directory should be something like:
\end{enumerate}

\begin{itemize}
\tightlist
\item
  ``C:/Users/YourName/Desktop/36-613/packages'' if you use Windows
\item
  ``/Users/YourName/Desktop/36-613/packages'' if you use Mac
\item
  ``\ldots{}'' if you are using the CMU cluster computers
\end{itemize}

\begin{enumerate}
\def\labelenumi{\alph{enumi}.}
\setcounter{enumi}{1}
\item
  In a code block, store the filepath in an object called
  \texttt{package\_path},
  e.g.~\texttt{package\_path\ \textless{}-\ "/Users/YourName/Desktop/36-613/packages"}.
  Repeat this at the command line / Console as well.
\item
  In the same code block, include the following line of code:
  \texttt{.libPaths(c(package\_path,\ .libPaths()))}
\item
  At the command line / Console (NOT in a code block), type
  \texttt{install.packages("tidyverse",\ lib\ =\ package\_path)}. This
  should install the \texttt{tidyverse} package.
\item
  If you successfully installed the \texttt{tidyverse} library, try
  running the \texttt{library(tidyverse)} code now. If you still run
  into any issues, please post to Piazza describing your issue and we
  will help you.
\end{enumerate}

\hypertarget{optional-formating-text-within-rstudio}{%
\section{OPTIONAL: Formating Text within
RStudio}\label{optional-formating-text-within-rstudio}}

There are a lot of ways to format text within RStudio, e.g.,
\emph{italics} and \textbf{bold} (just look at the .Rmd file to see how
I did this).
\href{https://www.rstudio.com/wp-content/uploads/2015/02/rmarkdown-cheatsheet.pdf}{See
here} for more tips/tricks on how to format things in R Markdown. As
you'll see throughout this class (and especially your project at the end
of the semester), well-formatted .html files can be a great way to
showcase data science results to the public online.

\hypertarget{optional-customizing-the-rstudio-user-interface}{%
\section{OPTIONAL: Customizing the RStudio User
Interface}\label{optional-customizing-the-rstudio-user-interface}}

Within RStudio there are several panes that contain various things
(Console, Help, Environment, History, Plots, etc). Here we discuss how
you can customize how these panes are displayed.

If you're using Mac, go to RStudio / Preferences / Pane Layout. If
you're using Windows, go to Tools / Global Options. Change the menu
options to arrange the panes as you see fit. Click Apply and OK.

Now (still within the RStudio / Preferences menu), click Appearance and
choose an appropriate font, font size, and theme. Click Apply and OK.
Minimizing the bottom-left and bottom-right panes is a nice trick, which
gives more vertical space to see your code and the output it's
generating. (Minimize/maximize buttons are in the top-right of each
pane.)

\hypertarget{optional-additional-customization-advice}{%
\section{OPTIONAL: Additional Customization
Advice}\label{optional-additional-customization-advice}}

\begin{itemize}
\tightlist
\item
  Under Preferences / Code / Display, you might consider adding the
  margin column and setting it to 80 characters, since most style guides
  suggest that you should keep lines of code at 80 characters or less
  when possible.
\item
  You can set your background color, font, font size, etc. under
  Preferences / Appearance. (I use Cobalt). ``Dark displays'' are often
  easier on the eyes and are environmentally-friendly in that they
  conserve energy on devices. Of course, this is strictly a personal
  preference, and people can get unnecessarily dogmatic about it, so you
  should just choose something that you like.
\item
  Under Preferences / Packages, you can opt to change your CRAN mirror
  to the ``Global (CDN) - RStudio'' option, as it is very reliable.
\item
  Under Preferences / Git / SVN, you can configure your version control
  preferences. People often recommend using Git with RStudio for version
  control purposes. The interface is easy to use even if you're a
  beginner programmer or Git user.
  \href{https://support.rstudio.com/hc/en-us/articles/200532077-Version-Control-with-Git-and-SVN}{This
  link} and
  \href{https://jennybc.github.io/2014-05-12-ubc/ubc-r/session03_git.html}{this
  link} have more information on this, if you're interested.
\end{itemize}

\hypertarget{optional-r-primers-on-rstudio-cloud}{%
\section{\texorpdfstring{OPTIONAL: \texttt{R} Primers on RStudio
Cloud}{OPTIONAL: R Primers on RStudio Cloud}}\label{optional-r-primers-on-rstudio-cloud}}

If you are struggling to install \texttt{R} and Rstudio on your
computer, and/or having difficulties with installing the
\texttt{tidyverse} then you \textbf{should make a free RStudio Cloud
account at
\href{https://rstudio.cloud/}{\texttt{https://rstudio.cloud/}}.} This is
a free, browser-based version of \texttt{R} and RStudio that also
provides access to a growing number of \texttt{R} tutorials / primers
relevant to this course.

After you create a RStudio Cloud account, click on the navigation menu
by ``Your Workspace''. Then click on
\href{https://rstudio.cloud/learn/primers}{``Primers''} to bring up a
menu of tutorials, with code primers you can choose to work through.
RStudio Cloud is a great practical alternative to use \textbf{in case we
are unable to resolve errors with regards to installation on your own
personal computer} (an unlikely scenario). We strongly encourage you to
use an installed version of \texttt{R} and RStudio throughout the
course, due to RStudio Cloud data limitations that are important for
your projects at the end of the semester.

\end{document}
